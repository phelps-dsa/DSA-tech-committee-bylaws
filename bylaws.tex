\documentclass[12pt,letter,twocolumn,oneside,draft]{article}

\usepackage{xcolor}
\usepackage[T1]{fontenc}
\usepackage{caption}

\newcommand{\note}[1]{\textcolor{purple}{#1}}
\newcommand{\committeename}{\emph{Chicago Democratic Socialists of America Technology and Information Security Committee}}
\newcommand{\cname}{\emph{CDSA Tech Committee}}
\newcommand{\cdsa}{\emph{Chicago Democratic Socialists of America}}

\begin{document}

\title{Bylaws of the \committeename{}}
\date{\today}
\maketitle



\tableofcontents



\setcounter{secnumdepth}{5}

\section{Name}

\paragraph{}
The \committeename{} (also known as the \emph{CDSA Tech and Info-Sec Committee}
or the \cname{}) is a sub-committee of the \emph{Executive Committee} of the
\emph{Chicago Democratic Socialists of America}.


\section{Purpose}

\paragraph{}
The \cname{} seeks to promote appropriate technological solutions to problems
while keeping to Democratic Socialist values, and to give voice to the struggle
of our fellow workers in the tech and tech-adjacent industries. To fulfil this
purpose the \cname{} will do the following:

\begin{enumerate}
    \item{Organize interested \cdsa{} members to provide design,
        implementation, and support for technical projects desired by the
        \cdsa{}, its committees, working groups, and members.}
    \item{Work together with local technology-focused groups and
        technology-focused groups in the Democratic Socialists of America
        at-large to share resources and achievements to further our Democratic
        Socialist values.}
    \item{Provide a platform for a Democratic Socialist critique of the tech
        industry which focuses on amplifying of the voices of workers and
        taking actions in solidarity with those workers.}
\end{enumerate}


\section{Membership}

\paragraph{Eligibility}
Any member of the \cdsa{} in good standing is eligible for membership in the
\cname{}.

\paragraph{Becoming a Member}
To become a Member of the committee, a prospective member must declare their
intent to join, and attend a minimum of three official meetings of the
\cname{}, within the span of six months. A prospective member becomes a Member
of the \cname{} at the start of the third required meeting and thus has the
full privileges of a Member at that time.

\paragraph{}
The \cname{} shall keep a list of all Members, prospective members, and their
status in the \cname{}.

\paragraph{Initial Members}
This initial list of Members of the \cname{} will consist of those individuals
who ratify this document.

\subsection{Member Conduct}

\paragraph{}
Members of the \cname{} shall comply with the \cdsa{} Harassment
Policy\footnote{See ADDENDUM 2 of the \emph{Constitution and By-Laws of the
Chicago Democratic Socialists of America} dated June 2018}. This includes any
and all communication both in-person and through digital channels such as chat
and email.

\paragraph{}
The \cname{} will maintain a system within itself to aid \cname{} Members in
the \cdsa{} Harassment Reporting process\footnote{See ADDENDUM 2, Section 2
\emph{Reporting Harassment} of the \emph{Constitution and By-Laws of the
Chicago Democratic Socialists of America} dated June 2018} as well as to
address matters more urgent than the official \cdsa{} Harassment Reporting
process can handle.

\subsection{Ending Membership}

\paragraph{Resignation}
A Member of \cname{} may resign at any time with any amount of notice by
notifying the \cname{} through any recognized communication channel.

\paragraph{Expiring Membership}
A membership may expire if a given Member does not attend any meeting of the
\cname{} for 365 days and a quorum of Members consent at a meeting.

\paragraph{Expulsion}
Should a Member of the \cname{} be found in violation of policies governing the
conduct of Members, details of the situation will be passed to the \cdsa{}
Harassment and Grievance Officers whose recommendation will be followed by the
\cname{}.


\section{Sub-Committees}

\paragraph{}
The \cname{} has the right to sub-organize itself into smaller groups
containing a non-zero subset of Members around a common theme called
Sub-Committees.

\paragraph{}
These Sub-Committees shall be governed by the same mechanisms as are defined in
this document to govern the \cname{} as a whole with the following exceptions
\begin{enumerate}
    \item{To become a member of a Sub-Committee, a Member of the \cname{} must
        declare their desire to be a member of the Sub-Committee at a meeting
        of that Sub-Committee and need not go through the same membership
        process as described in this document\footnote{Section 6
        \emph{Members}.}}
    \item{Sub-Committees are not required to maintain their own Harassment and
        Grievance process. They will rely on the Harassment and Grievance
        process of the \cname{}\footnote{Section 3 Subsection 1 Subsubsection 0
        Paragraph 2}.}
    \item{Sub-Committees do not require the \cdsa{} Harassment and Grievance
        Officers' involvement to expel a member from the Sub-Committee, they
        must use the Harassment and Grievance process of the \cname{}.}
\end{enumerate}


\section{Responsibilities}

\paragraph{}
The \cname{} will maintain a list of Responsibilities each of which
encapsulates a discrete set of work which is necessary to carry out the
business of the \cname{}.

\paragraph{}
Each Responsibility must be assigned to a Member or Sub-Committee, hereafter
called a Syndic, of the \cname{} who is then required to ensure that the
\cname{} enacts the content of the assigned Responsibility. The mechanism to
assign a Responsibility to a Syndic must be defined in said Responsibility at
the time of its creation and may include voting to assign a Responsibility to a
Syndic, assigning the Responsibility to a Syndic by a lottery, or allowing a
Responsibility to self-assigned by a Member volunteering. Regardless of system
of assignment, the assignee must consent to the assignment.

\paragraph{}
The \cname{} must make efforts to ensure that Responsibilities are distributed
equally and fairly across as diverse a set of backgrounds as available to the
\cname{}.  Specifically, the distribution should be sure to reduce the number
of Responsibilities assigned to cisgender white men when alternatives are
available. No less than once every six months, the \cname{} will conduct an
audit of it's distribution of Responsibilities to ensure it is fulfilling this
requirement.


\section{Meetings}

\paragraph{Attendee Conduct}
Attendees of Meetings of the \cname{} are expected to conduct themselves in
accordance with the \emph{Member Conduct} section of this
document\footnote{Section 3 \emph{Membership}, Subsection 1 \emph{Member
Conduct}} as well as the guidelines defined by \emph{Feminist
Process}\footnote{See \emph{ADDENDUM 1: FEMINIST PROCESS} of the
\emph{Constitution and By-Laws of the Chicago Democratic Socialists of America}
dated June 2018}

\subsection{Scheduling and Locations}

\paragraph{}
The committee must hold at minimum one meeting per month at a \cname{} approved
physical location or digital meeting space. Announcements of the meeting time
and place must be sent out no later than two weeks prior to the meeting taking
place to give adequate notice.

\paragraph{}
Meeting locations, times, and mediums shall be decided by a decision consented
to by the \cname{}. This power may be delegated within a Responsibility and
assigned to a Syndic by consent of the \cname{}.

\paragraph{Accessibility}
\note{This section still needs work. There are several DSA chapters with
guidelines for this and we should copy those.}

\subsection{The Meeting Facilitator}

\paragraph{}
The Meeting Facilitator, or simply Facilitator, ensures that a meeting is
running smoothly, ethically, safely, and fairly. They are charged with tracking
the state of a decision and mediating discussion of agenda items between
attendees of the meeting to ensure that consensus can be reached in a quick but
responsible manner.

\paragraph{}
There may be up to two Facilitators for a given meeting. If there are two, one
must not be a cisgender white man.

\paragraph{} 
The Member or Members assigned the role of Facilitator changes with each
meeting of the \cname{}. Members in attendance may volunteer for the position
at the start of the meeting. For each volunteer, given that the volunteer
fulfils the requirements for the position, and the members in attendance
consent without debate, that member become a Facilitator for the duration of
the meeting. Each volunteer should be voted on individually in random order
until there are two Facilitators or there are no more volunteers to consider.
If after going through the volunteers no Facilitator has been established, the
role will be assigned by random lottery of the remaining attendees of the
meeting to a single Member. 

\subsection{The Composition of a Meeting}

\paragraph{}
This subsection describes the structure and flow of a meeting of the \cname{}.

\subsubsection{Commencement}

\paragraph{}
A meeting commences at its predefined starting time at which time the members
in attendance must either consent to start the meeting or delay by a defined period of
time. 

\paragraph{}
After starting the meeting, the attendees will select one or two
Facilitators\footnote{See Section 6 \emph{Meetings}, Subsection 2 \emph{The
Meeting Facilitator}}. Once selected, the Meeting Facilitator(s) will proceed
to call for consensus on approval of the agenda for the meeting. If the agenda
is not consented to, then the meeting ends. The Facilitator(s) will then begin
to mediate discussion through each agenda item.

\subsubsection{Agenda Items}

\paragraph{}
For each item on the agenda, discussion will cycle around giving each attendee
the opportunity to give input on the item being discussed. It is up to the
Facilitator(s) to ensure that each attendee who wishes to speak is given the
opportunity to do so, and that the tenets set forth in Feminist Process are
being respected. Conversation will cycle through the attendees until they
consent to, if applicable, vote on the matter being discussed, or proceed to
the next item on the agenda.

\paragraph{}
As conversation cycles through the room, the Facilitator(s) has the right to
interrupt any member at anytime in order to perform some aspect of their
responsibilities as Facilitators, but may not interrupt to participate directly
in the conversation at hand. The Facilitator must wait for their turn in the
rotation like all other participants in the discussion.

\subsubsection{Conclusion}

\paragraph{}
Once the final agenda item as been resolved, the meeting concludes.

\subsection{Emergency Meetings}

\paragraph{}
If Members become aware of an urgent and important matter which they believe
requires the attention of the \cname{}, an Emergency Meeting of the \cname{}
must be organized with at least 24 hours prior notice. 

\paragraph{}
The proposers of an Emergency Meeting must provide an agenda to guide the
Emergency Meeting and it must be acknowledged by a quorum of members at least
four hours prior to the expected start time in order to be considered a valid
meeting.  This agenda cannot be changed after it's proposal and must be
accepted or rejected as it was presented. Only the decision items that are
listed on the emergency meeting agenda can be voted on and no further items may
be added after the agenda is proposed. 

\paragraph{}
No amendments to the bylaws (this document) may be made during an Emergency
Meeting.

\paragraph{}
Emergency Meetings cannot require all attendees to co-locate into the same
physical space. A remote attendance option through an approved communication
medium must be available. 

\paragraph{}
Except for alterations described above in this subsection, Emergency Meetings
are governed by the same rules as standard meetings\footnote{See Section 6
\emph{Meetings}}.


\section{Governance of the Committee}

\paragraph{}
The \cname{} shall be governed by a consensus based decision-making process
enacted by a quorum of Members at recognized meetings of the \cname{}. In order
for the \cname{} to take an action it must be consented to by the Members of
the \cname{} through the decision making process defined in this section.

\paragraph{Quorum}
To make decisions, a quorum of two-thirds the average attendance of the
previous three meetings of the \cname{} must be reached. For the first three
meetings under these bylaws, which will occur without a previous three meetings
from which to determine quorum, quorum shall be determined by using the filler
value of five for the attendance of any unheld meeting until the fourth meeting
when the standard rule for quorum can be used.

\subsection{Decisions}

\paragraph{Decision Making}
In order for the \cname{} to make a decision, the item being discussed must
have consent from all Members in attendance less one for each multiple of
quorum in attendance rounded down. For example, if quorum for a meeting is five
and eight Members are in attendance, the consent of seven Members is required
to make a decision\footnote{See Appendix A \emph{Voting and Quorum} for more information}.

\paragraph{}
Decisions must result in a new Responsibility or a replacement for an existing
Responsibility in order to ensure that decisions are actionable and the
\cname{} is accountable for them.


\section{Amendments}

\paragraph{}
This document may be amended by the usual decision making process with the
exception that quorum for the decision is set as the higher of either 1/3 of
the \cname{}'s total membership or what the established quorum for the meeting
would be normally.

\pagebreak

\appendix

\onecolumn
\section{Voting and Quorum}

To determine the number of consenting Members necessary to make a decision, the
equation which must be used is
\( v_{n} = m_{a} - \lfloor \frac{m_{a}}{m_{q}} \rfloor \),
where $v_{n}$ is the number of consenting Members needed to make a decision,
$m_{a}$ is the number of Members in attendance of the meeting, and $m_{q}$ is
the number of Members needed to achieve quorum. Some examples are listed in
\emph{Table 1} below.

\captionof{table}{Example Meeting Attendance and Vote Requirements} \label{tab:title} 
\begin{tabular}{| c | c | c |}
    \hline
    $m_{q}$ & $m_{a}$ & $v_{n}$ \\
    \hline
    5 & 8 & 7 \\
    \hline
    5 & 11 & 9 \\
    \hline
\end{tabular}
\end{document}
